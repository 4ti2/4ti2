%%%%%%%%%%%%%%%%%%%%%%%%%%%%%%%%%%%%%%%%%%%%%%%%%%%%%%%%%%%%%%%%%%%
%
% 4ti2 User Guide, Beginner's guide
%
% $Id$
%
%%%%%%%%%%%%%%%%%%%%%%%%%%%%%%%%%%%%%%%%%%%%%%%%%%%%%%%%%%%%%%%%%%%

\chapter{Beginner's guide}

%\FourTiTwo{} is a software package that is specifically tailored for
%certain computational problems on linear spaces and integer
%lattices.
In this part, we use a few sample problems to introduce
you to the basic functionality of \FourTiTwo{}. After working
through this part, you should know about \emph{linear systems} and
should be able to do computations using the following functions:
\begin{itemize}
\item \File{qsolve}, \File{rays}, \File{circuits}
\item \File{zsolve}, \File{hilbert}, \File{graver}, \File{ppi}
\item \File{markov}, \File{groebner}, \File{normalform}, \File{minimize}
\end{itemize}



\section{Linear systems and their encodings}

\subsection{Linear systems and integer linear systems}

In \FourTiTwo, a \emph{linear system} is defined by $d$ constraints
$Ax\sim b$ in $n$ unknowns $x$, where each constraint is either
$\leq$, $=$ or $\geq$, that is $\sim\;\in\{\leq,=,\geq\}^d$.
Moreover, one may specify sign constraints on the variables that
need to be respected in an explicit representation of all solutions.

There is no particular difference in \FourTiTwo{} between a linear
system and an integer linear system. Currently, the user chooses
between one of the two by calling the appropriate functions on the
linear system.

\subsection{Specifying a linear system}

In order to use a linear system as input, one needs to specify its
parts to \FourTiTwo. As our running example, take
\[
\left(
\begin{array}{cccc}
1 & 1 & 1 & 1\\
1 & 2 & 3 & 4\\
\end{array}
\right)x
\begin{array}{c}
\leq\\
\leq\\
\end{array}
\left(
\begin{array}{c}
6\\
10\\
\end{array}
\right)
\]
with sign constraints $(1,2,2,0)$, which we will explain below.

First, one has to give his problem a project name, say
\FILE{PROJECT}.
\begin{itemize}
\item The matrix $A$ has to be put into the main project file, say
\File{PROJECT.mat}.
\begin{myverbatim}
2 4
1 1 1 1
1 2 3 4
\end{myverbatim}
\item The relations $\sim$ then have to be specified in \File{PROJECT.rel}.
\begin{myverbatim}
1 2
< <
\end{myverbatim}
\item The right-hand side vector goes into \File{PROJECT.rhs}.
\begin{myverbatim}
1 2
6 10
\end{myverbatim}
\item And finally, the sign constraints end up in \File{PROJECT.sign}.
\begin{myverbatim}
1 4
1 2 2 0
\end{myverbatim}
\end{itemize}

{\bf Note.}
\begin{itemize}
\item The input files all have the format of a matrix, preceded by
  the matrix dimensions. As the dimensions specify how many symbols
  have to be read, the matrix could be given in only one line or even
  in many lines of different lengths.
\item In \FourTiTwo{} version 1.3, all appearing numbers have to be
  integers.
\item Consequently, this implies that at the moment, \File{qsolve}
  only supports homogeneous linear systems, that is systems with $b=0$.
  Minimal inhomogeneous solutions could have rational components.
\end{itemize}

If the system is solved over $\R$ (using \File{qsolve}),
\FourTiTwo{} returns two sets of integer vectors:
\begin{itemize}
\item a set $H$ of support minimal homogeneous solutions, and
\item a set $F$ defining the linear vector space the solution set
lives in.
\end{itemize}
As only \emph{homogeneous} linear systems are supported in this
version of \FourTiTwo, no list of minimal inhomogeneous solutions is
computed. Any solution $z$ of the linear system can now be written
as
\begin{equation}\label{Equation: Continuous representation}
z=\sum \alpha_j h_j+ \sum \beta_k f_k
\end{equation}
with $h_j\in H$, $f_k\in F$, and $\alpha_j\geq 0$.

If the system is solved over $\Z$ (using \File{zsolve}),
\FourTiTwo{} returns three sets of integer vectors:
\begin{itemize}
\item a set $H$ of support minimal homogeneous \emph{integer} solutions,
\item a set $I$ of support minimal inhomogeneous \emph{integer}
solutions, and
\item a set $F$ defining the sublattice of $\Z^n$ the solution set
lives in.
\end{itemize}
Any solution $z$ of the linear system can now be written as
\begin{equation}\label{Equation: Integer representation}
z=i+ \sum \alpha_j h_j+\sum \beta_k f_k
\end{equation}
for some $i\in I$ and with $h_j\in H$, $f_j\in F$, and
$\alpha_j\in\Z_+0$.

{\bf Sign file.} Let us finally clarify what the sign file
\File{proj.sign} is good for. The sign file may declare a variable
to be nonnegative ($1$), to be non-positive ($-1$), or to consider
both cases independently and unite the answers ($2$). If a nonzero
sign has been assigned to a variable, the explicit representations
(\ref{Equation: Continuous representation}) and (\ref{Equation:
Integer representation}) above of a solution $z$ have to respect the
sign on that component. The default setting for each variable is
$0$, that is, the sign need not be respected in the explicit
representation. In our example above, the first variable is declared
to be nonnegative, the second and the third one expand to $2\cdot
2=4$ orthant constraints, and the fourth variable is unconstrained.
Note, however, that \FourTiTwo{} does \emph{not} decompose the
problem internally into the four problems with sign patterns
$(1,1,1,0)$, $(1,1,-1,0)$, $(1,-1,1,0)$, and $(1,-1,-1,0)$, but
deals with them more efficiently at the same time.
