%%%%%%%%%%%%%%%%%%%%%%%%%%%%%%%%%%%%%%%%%%%%%%%%%%%%%%%%%%%%%%%%%%%
%
% 4ti2 User Guide, Advanced guide
%
% $Id$
%
%%%%%%%%%%%%%%%%%%%%%%%%%%%%%%%%%%%%%%%%%%%%%%%%%%%%%%%%%%%%%%%%%%%

\chapter{Advanced guide}

In this part, we deal with several more advanced problem
specifications in \FourTiTwo{}.

First we introduce \emph{affine systems} and their encodings. In
fact, affine systems are the basic objects used in \FourTiTwo{},
since every linear system is transformed into an affine system.
However, in the integer situation, it is not always possible to
transform an affine system back into a linear system without adding
variables or modulo constraints.

Next, we demonstrate how one can exploit bounds on integer variables
to truncate the solution set.

Again, we use a few sample problems to demonstrate how to use
\FourTiTwo{}. After working through this part, you should know about
how to run the following functions on affine systems:
\begin{itemize}
\item \File{qsolve}, \File{rays}, \File{circuits}
\item \File{zsolve}, \File{hilbert}, \File{graver}, \File{ppi}
\item \File{markov}, \File{groebner}, \File{normalform}, \File{minimize}
\end{itemize}


\section{Affine systems and their encodings}

In \FourTiTwo, the definition of an \emph{affine system} is slightly
different for the continuous and the integer situation. In fact, the
definition for the integer case is simply the integer analogue to
the definition for the continuous case.

\subsection{Continuous affine systems}

Let $a+\Lattice$ be a linear affine space given by the vector $a$
and by generators for the linear space $\Lattice$. We wish to find a
finite sign-compatible description for the set of all vectors $x\in
a+\Lattice$.

\subsection{Integer affine systems}

Let $a+\Lattice$ be an ``integer linear affine space'' given by the
vector $a\in\Z^n$ and by generators for the lattice
$\Lattice\subseteq\Z^n$. We wish to find a finite sign-compatible
description for the set of all (integer) vectors $x\in a+\Lattice$.

\subsection{Specifying an affine system in \FourTiTwo{}}

Currently, only homogeneous affine systems can be solved in
\FourTiTwo{} over $\R$ and over $\Z$ using the functions
\File{qsolve} and \File{zsolve}, respectively. In order to call
these functions, one needs to specify an affine system to
\FourTiTwo.

As an example, take the homogeneous part of our running example
above
\[
\left(
\begin{array}{cccc}
1 & 1 & 1 & 1\\
1 & 2 & 3 & 4\\
\end{array}
\right)x
\begin{array}{c}
\leq\\
\leq\\
\end{array}
\left(
\begin{array}{c}
0\\
0\\
\end{array}
\right)
\]
with sign constraints $(1,2,2,0)$, and rewrite it as an affine
system. As the linear subspace and the lattice $\Lattice$,
respectively, we are dealing with the continuous and the integer
kernel of the $2\times 4$ problem matrix. The following two vectors
generate both the linear space over $\R$ and the lattice over $\Z$:
\[
\left(
\begin{array}{r}
1\\
-2\\
1\\
0\\
\end{array}
\right),\;\;\; \left(
\begin{array}{r}
2\\
-3\\
0\\
1\\
\end{array}
\right)
\]
Thus, we are looking for a finite sign-compatible description for
all $x$ that can be written as
\[
x=\left(
\begin{array}{rr}
1 & 2\\
-2 & -3\\
1 & 0\\
0 & 1\\
\end{array}
\right)\lambda,
\]
with $\lambda\in\R^2$ and $\lambda\in\Z^2$, respectively.

In order to solve this affine system using \File{qsolve} or
\File{zsolve}, we have

\begin{itemize}
\item The generators of $\Lattice$ have to be put into the file \File{PROJECT.lat}.
\begin{myverbatim}
2 4
1 -1 1 0
2 -3 0 1
\end{myverbatim}
\item The sign constraints end up in \File{PROJECT.sign}.
\begin{myverbatim}
1 4
1 2 2 0
\end{myverbatim}
\end{itemize}
