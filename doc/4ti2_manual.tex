%%%%%%%%%%%%%%%%%%%%%%%%%%%%%%%%%%%%%%%%%%%%%%%%%%%%%%%%%%%%%%%%%%%
%
% 4ti2 User Guide
%
% $Id$
%
%%%%%%%%%%%%%%%%%%%%%%%%%%%%%%%%%%%%%%%%%%%%%%%%%%%%%%%%%%%%%%%%%%%
\documentclass[12pt,a4]{book}
%%%%%%%%%%%%%%%%%%%%%%%%%%%%
\usepackage{amssymb}
\usepackage{amsmath}
\usepackage{amsfonts}
\usepackage{rotating}
\usepackage{myfancyheadings}
\usepackage{color}
\usepackage{verbatim}
\usepackage{epic}
\usepackage{enumerate}
\usepackage{booktabs}
%\usepackage{hyperref}
\usepackage{graphics}

\textwidth15cm
\textheight21cm
\oddsidemargin0.7cm
\evensidemargin0.7cm
\parindent0cm
\linespread{1.2}\normalsize

\frenchspacing
\parskip=1ex plus .25ex minus .25ex

\pagestyle{fancy} \rhead{\rm\thepage} \cfoot{}

\newtheorem{theorem}{Theorem}[section]
\newtheorem{prop}[theorem]{Proposition}
\newtheorem{cor}[theorem]{Corollary}
\newtheorem{Call}{Call}
\newtheorem{Result}{Result}
\newtheorem{note}{Note}

\newcommand{\boprob}{\noindent {\bf Problem. }}
\newcommand{\boremark}{\noindent {\bf Remark. }}
\newcommand{\boexample}{\noindent {\bf Example. }}
\newcommand{\boproof}{\noindent {\bf Proof. }}
\newcommand{\eoproof}{\hspace*{\fill} $\square$ \vspace{5pt}}
\newcommand{\red}{\sqsubseteq}
\newcommand{\rhs}{right-hand side}
\newcommand{\nred}{\sqsubseteq \hspace{-8pt} | \hspace{5pt}}
\newcommand{\framewidth}{15.5cm}
\newcommand{\psp}{{positive sum property} }
\newcommand{\pspwrt}{{positive sum property with respect to} }
\newcommand{\C}{\mathbb C}
\newcommand{\R}{\mathbb R}
\newcommand{\Q}{\mathbb Q}
\newcommand{\Z}{\mathbb Z}
\newcommand{\N}{\mathbb N}
\newcommand{\X}{\mathbb X}
\newcommand{\E}{\mathbb E}
%\def \E { {\bbold E} }
\newcommand{\1}{{\bf 1}}
\newcommand{\Orthant}{\mathbb O}
\newcommand{\PP}{{({\text{P}})}}
\newcommand{\LP}{{({\text{LP}})}}
\newcommand{\IP}{{({\text{IP}})}}
\newcommand{\AIP}{{({\text{AIP}})}}
\newcommand{\MOD}{{({\text{MOD}})}}
\newcommand{\CPP}{{({\text{CIP}})}}
\newcommand{\GLP}{{{\cal G}_{\text{LP}}}}
\newcommand{\GIP}{{{\cal G}_{\text{IP}}}}
\newcommand{\GCPP}{{{\cal H}_{\text{CIP}}}}
\newcommand{\hilbert}{{\cal H}}
\newcommand{\ind}{\mbox{ind}}
\newcommand{\co}{\mbox{cone}}
\newcommand{\SA}{{{\cal S}_\Lattice}}
\newcommand{\symGIP}{{\cal G}^{\text{sym}}}
\newcommand{\diag}{\mbox{diag}}
\newcommand{\sigmabar}{{\bar{\sigma}}}
\newcommand{\taubar}{{\bar{\tau}}}
\newcommand{\Lattice}{\mathcal{L}}
\newcommand{\Feasible}{\mathcal{F}}
\newcommand{\ExtFeasible}{\mathcal{E}}
\newcommand{\Graph}{\mathcal{G}}
\newcommand{\st}{\mathrm{s.t.}}

\newcommand{\Kw}[1]{\underline{#1}}
\newcommand{\keyw}[1]{{\bf #1}}
\newcommand{\FourTiTwo}{{\tt 4ti2}}
\newcommand{\important}{\textbf}

\newcommand{\File}[1]{{\tt #1}}

%-- Create a verbatim environment which indents itself.
\newenvironment{myverbatim}%
  {\quote\verbatim}%
  {\endverbatim\endquote}

\DeclareMathOperator{\supp}{supp} \DeclareMathOperator{\conv}{conv}
\DeclareMathOperator{\cost}{cost} \DeclareMathOperator{\cone}{cone}
\DeclareMathOperator{\linspan}{span}
\DeclareMathOperator{\rank}{rank}
\DeclareMathOperator{\monoid}{monoid}

\begin{document}

\thispagestyle{empty}

\title{User's Guide for \FourTiTwo{} version 1.3.1}

\author{A software package\\ for algebraic,
geometric and combinatorial problems on linear spaces}
\date{\today}

\maketitle

\newpage
\thispagestyle{empty}

%\frontmatter

\tableofcontents

%\mainmatter

%%%%%%%%%%%%%%%%%%%%%%%%%%%%%%%%%%%%%%%%%%%%%%%%%%%%%%%%%%%%%%%%%%%
%
% 4ti2 User Guide, Beginner's guide
%
% $Id$
%
%%%%%%%%%%%%%%%%%%%%%%%%%%%%%%%%%%%%%%%%%%%%%%%%%%%%%%%%%%%%%%%%%%%

\chapter{Beginner's guide}

%\FourTiTwo{} is a software package that is specifically tailored for
%certain computational problems on linear spaces and integer
%lattices.
In this part, we use a few sample problems to introduce
you to the basic functionality of \FourTiTwo{}. After working
through this part, you should know about \emph{linear systems} and
should be able to do computations using the following functions:
\begin{itemize}
\item \File{qsolve}, \File{rays}, \File{circuits}
\item \File{zsolve}, \File{hilbert}, \File{graver}, \File{ppi}
\item \File{markov}, \File{groebner}, \File{normalform}, \File{minimize}
\end{itemize}



\section{Linear systems and their encodings}

\subsection{Linear systems and integer linear systems}

In \FourTiTwo, a \emph{linear system} is defined by $d$ constraints
$Ax\sim b$ in $n$ unknowns $x$, where each constraint is either
$\leq$, $=$ or $\geq$, that is $\sim\;\in\{\leq,=,\geq\}^d$.
Moreover, one may specify sign constraints on the variables that
need to be respected in an explicit representation of all solutions.

There is no particular difference in \FourTiTwo{} between a linear
system and an integer linear system. Currently, the user chooses
between one of the two by calling the appropriate functions on the
linear system.

\subsection{Specifying a linear system in \FourTiTwo}

In order to use a linear system as input, one needs to specify its
parts to \FourTiTwo. As our running example, take
\[
\left(
\begin{array}{cccc}
1 & 1 & 1 & 1\\
1 & 2 & 3 & 4\\
\end{array}
\right)x
\begin{array}{c}
\leq\\
\leq\\
\end{array}
\left(
\begin{array}{c}
6\\
10\\
\end{array}
\right)
\]
with sign constraints $(1,2,2,0)$, which we will explain below.

First, one has to give his problem a project name, say
\File{PROJECT}.
\begin{itemize}
\item The matrix $A$ has to be put into the main project file, say
\File{PROJECT.mat}.
\begin{myverbatim}
2 4
1 1 1 1
1 2 3 4
\end{myverbatim}
\item The relations $\sim$ then have to be specified in \File{PROJECT.rel}.
\begin{myverbatim}
1 2
< <
\end{myverbatim}
\item The right-hand side vector goes into \File{PROJECT.rhs}.
\begin{myverbatim}
1 2
6 10
\end{myverbatim}
\item And finally, the sign constraints end up in \File{PROJECT.sign}.
\begin{myverbatim}
1 4
1 2 2 0
\end{myverbatim}
\end{itemize}

{\bf Note.}
\begin{itemize}
\item The input files all have the format of a matrix, preceded by
  the matrix dimensions. As the dimensions specify how many symbols
  have to be read, the matrix could be given in only one line or even
  in many lines of different lengths.
\item In \FourTiTwo{} version 1.3, all appearing numbers have to be
  integers.
\item Consequently, this implies that at the moment, \File{qsolve}
  only supports homogeneous linear systems, that is systems with $b=0$.
  Minimal inhomogeneous solutions could have rational components.
\end{itemize}

If the system is solved over $\R$ (using \File{qsolve}),
\FourTiTwo{} returns two sets of integer vectors:
\begin{itemize}
\item a set $H$ of support minimal homogeneous solutions, and
\item a set $F$ defining the linear vector space the solution set
lives in.
\end{itemize}
As only \emph{homogeneous} linear systems are supported in this
version of \FourTiTwo, no list of minimal inhomogeneous solutions is
computed. Any solution $z$ of the linear system can now be written
as
\begin{equation}\label{Equation: Continuous representation}
z=\sum \alpha_j h_j+ \sum \beta_k f_k
\end{equation}
with $h_j\in H$, $f_k\in F$, and $\alpha_j\geq 0$.

If the system is solved over $\Z$ (using \File{zsolve}),
\FourTiTwo{} returns three sets of integer vectors:
\begin{itemize}
\item a set $H$ of support minimal homogeneous \emph{integer} solutions,
\item a set $I$ of support minimal inhomogeneous \emph{integer}
solutions, and
\item a set $F$ defining the sublattice of $\Z^n$ the solution set
lives in.
\end{itemize}
Any solution $z$ of the linear system can now be written as
\begin{equation}\label{Equation: Integer representation}
z=i+ \sum \alpha_j h_j+\sum \beta_k f_k
\end{equation}
for some $i\in I$ and with $h_j\in H$, $f_j\in F$, and
$\alpha_j\in\Z_+0$.

{\bf Sign file.} Let us finally clarify what the sign file
\File{proj.sign} is good for. The sign file may declare a variable
to be nonnegative ($1$), to be non-positive ($-1$), or to consider
both cases independently and unite the answers ($2$). If a nonzero
sign has been assigned to a variable, the explicit representations
(\ref{Equation: Continuous representation}) and (\ref{Equation:
Integer representation}) above of a solution $z$ have to respect the
sign on that component. The default setting for each variable is
$0$, that is, the sign need not be respected in the explicit
representation. In our example above, the first variable is declared
to be nonnegative, the second and the third one expand to $2\cdot
2=4$ orthant constraints, and the fourth variable is unconstrained.
Note, however, that \FourTiTwo{} does \emph{not} decompose the
problem internally into the four problems with sign patterns
$(1,1,1,0)$, $(1,1,-1,0)$, $(1,-1,1,0)$, and $(1,-1,-1,0)$, but
deals with them more efficiently at the same time.


\section{Brief tutorial}

\subsection{Solving linear systems over $\R$ and over $\Z$}

To solve the system
\[
  \begin{array}{rcrcrlcl}
    x & - & y & \leq & 2\\
  -3x & + & y & \leq & 1\\
    x & + & y & \geq & 1\\
      &   & y & \geq & 0\\
  \end{array}
\]
over $\R$, we have to create the files

\begin{center}
  \begin{tabular}{|l|l|l|l|}
\hline
    \text{ PROJECT.mat } & \text{ PROJECT.rel } & \text{ PROJECT.rhs } & \text{ PROJECT.sign }\\
\hline
  $\begin{array}{rrrr}& 3 & 2 & \\& 1 & -1\\& -3 & 1\\& 1 & 1 &\\ \end{array}$ & 
  $\begin{array}{rrrrr}& 1 & 3 & \\& < &  < & = & \\ \\ \\\end{array}$ & 
  $\begin{array}{rrrrr}& 1 & 3 & \\& 2 &  1 & 1 & \\ \\ \\\end{array}$ &
  $\begin{array}{rrrr}& 1 & 2 & \\& 0 &  1 & \\ \\ \\\end{array}$\\
\hline
  \end{tabular}
\end{center}

and then call

\begin{center}
{\tt ./qsolve PROJECT}
\end{center}

This call creates three files

\begin{center}
  \begin{tabular}{|l|l|l|}
\hline
    \text{ PROJECT.qinhom } & \text{ PROJECT.qhom } & \text{ PROJECT.qfree }\\
\hline
  $\begin{array}{rrrr}& 3 & 2 &\\& 0 & 1 &\\& 0 & 2 &\\& 1 & 0 &\\\end{array}$ & 
  $\begin{array}{rrrr}& 2 & 2 &\\& 1 & 1 &\\& 1 & 3 & \\ \\\end{array}$ &
  $\begin{array}{rrrr}& 0 & 2 & \\ \\ \\ \\ \end{array}$\\
\hline
  \end{tabular}
\end{center}

which correspond to the explicit description of all solutions:
\begin{center}
  \begin{tabular}{cc}
    \emph{Feasible solutions} & \emph{Computed representation}\\
       \input{feasible.pdf_t} & \input{solved_LP.pdf_t}    \\
  \end{tabular}
\end{center}
\[
\conv\left({1\choose 0},{2\choose 0},{0\choose 1}\right) + \cone\left({1\choose 1},{1\choose 3}\right).
\]
Note that in the pictures above, the three colored cones are only a
simplification and shall visualize the covering of the feasible
region by infinitely many shifted copies of the cone
\[
\cone\left({1\choose 1},{1\choose 2},{1\choose 3}\right),
\]
appended to each point in
\[
\conv\left({1\choose 0},{2\choose 0},{0\choose 1}\right).
\]

Let us now turn to the integer situation, that is, let us solve the system
\[
  \begin{array}{rcrcrlcl}
    x & - & y & \leq & 2\\
  -3x & + & y & \leq & 1\\
    x & + & y & \geq & 1\\
      &   & y & \geq & 0\\
  \end{array}
\]
over $\Z$. As the linear system itself is unchanged, we can use the
same the same files as above in order to specify the linear system.

\begin{center}
  \begin{tabular}{|l|l|l|l|}
\hline
    \text{ PROJECT.mat } & \text{ PROJECT.rel } & \text{ PROJECT.rhs } & \text{ PROJECT.sign }\\
\hline
  $\begin{array}{rrrr}& 3 & 2 & \\& 1 & -1\\& -3 & 1\\& 1 & 1 &\\ \end{array}$ & 
  $\begin{array}{rrrrr}& 1 & 3 & \\& < &  < & = & \\ \\ \\\end{array}$ & 
  $\begin{array}{rrrrr}& 1 & 3 & \\& 2 &  1 & 1 & \\ \\ \\\end{array}$ &
  $\begin{array}{rrrr}& 1 & 2 & \\& 0 &  1 & \\ \\ \\\end{array}$\\
\hline
  \end{tabular}
\end{center}

Then, however, we call

\begin{center}
{\tt ./zsolve PROJECT}
\end{center}

This call creates three files

\begin{center}
  \begin{tabular}{|l|l|l|}
\hline
    \text{ PROJECT.zinhom } & \text{ PROJECT.zhom } & \text{ PROJECT.zfree }\\
\hline
  $\begin{array}{rrrr}& 4 & 2 &\\& 0 & 1 &\\& 0 & 2 &\\& 1 & 0 &\\ & 1 & 1 &\\\end{array}$ & 
  $\begin{array}{rrrr}& 3 & 2 &\\& 1 & 1 &\\& 1 & 2 &\\& 1 & 3 & \\ \\\end{array}$ &
  $\begin{array}{rrrr}& 0 & 2 & \\ \\ \\ \\ \\ \end{array}$\\
\hline
  \end{tabular}
\end{center}

which correspond to the explicit description of all integer solutions:
\begin{center}
  \begin{tabular}{cc}
    \emph{Feasible solutions} & \emph{Computed representation}\\
       \input{feasible.pdf_t} & \input{solved_IP.pdf_t}    \\
  \end{tabular}
\end{center}
\[
\left\{{1\choose 0},{2\choose 0},{0\choose 1},{1\choose 1}\right\} + \monoid\left({1\choose 1},{1\choose 2},{1\choose 3}\right).
\]
Note that in the pictures above, we are only interested in the
\emph{lattice points} inside the colored regions! The full regions are
colored to visualize the covering of all feasible integer solutions by
finitely many shifted copies of the monoid 
\[
\monoid\left({1\choose 1},{1\choose 2},{1\choose 3}\right).
\]


\subsection{Computing extreme rays and Hilbert bases}

Let us consider the set of magic $3\times 3$ squares with nonnegative
real entries, that is, the set of all $3\times 3$ arrays with
non-negative real entries whose row sums, column sums, and main
diagonal sums all add up to the same number, the magic constant of the
square. 
\vspace{-0.3cm}
\begin{center}
  \input{magicsqs.pdf_t}
\end{center}
\vspace{-0.4cm}
Clearly, addition of two magic squares gives another magic square,
as well as does multiplication of a magic square by a nonnegative
number. Therefore, we may talk about the cone of magic $3\times 3$
squares. In fact, this cone is a pointed rational polyhedral cone
described by the linear system 
\begin{eqnarray*}
x_{11}+x_{12}+x_{13}
& = & x_{21}+x_{22}+x_{23}\\
& = & x_{31}+x_{32}+x_{33}\\
& = & x_{11}+x_{21}+x_{31}\\
& = & x_{12}+x_{22}+x_{32}\\
& = & x_{13}+x_{23}+x_{33}\\
& = & x_{11}+x_{22}+x_{33}\\
& = & x_{31}+x_{22}+x_{13}\\
&   & x_{ij} \geq 0,\;\;\; \text{ for all } i,j=1,2,3.
\end{eqnarray*}
Bringing all $x_{ij}$ to the left-hand side of these equations, the
matrix $A_{3\times 3}$ defining this linear system is
\[
A_{3\times 3}=\left(
\begin{array}{rrrrrrrrr}
1 & 1 & 1 & -1 & -1 & -1 &  0 &  0 &  0\\
1 & 1 & 1 &  0 &  0 &  0 & -1 & -1 & -1\\
0 & 1 & 1 & -1 &  0 &  0 & -1 &  0 &  0\\
1 & 0 & 1 &  0 & -1 &  0 &  0 & -1 &  0\\
1 & 1 & 0 &  0 &  0 & -1 &  0 &  0 & -1\\
0 & 1 & 1 &  0 & -1 &  0 &  0 &  0 & -1\\
1 & 1 & 0 &  0 & -1 &  0 & -1 &  0 &  0\\
\end{array}
\right).
\]
Below, we will deal with the more interesting case of \emph{integer} magic
squares. For the moment, however, we wish to compute the extreme rays
of the magic square cone $\{z:A_{3\times 3}z=0,z\geq 0\}$. 

In order to call the function \File{rays}, we are not allowed to
specify a right-hand side in \File{PROJECT.rhs}, as it must be
$0$. Moreover, we do not have to specify relations in
\File{PROJECT.rel} if there are equations \emph{only} (= default
setting for \File{rays}). Since we have nonnegativity constraints on
\emph{all} variables, we do not have to specify signs in
\File{PROJECT.sign} (= default setting for \File{rays}). Thus, we only
have to specify the problem matrix $A_{3\times 3}$.
\begin{center}
  \begin{tabular}{|l|}
\hline
  \text{ PROJECT.mat } \\
\hline
$\begin{array}{rrrrrrrrrrr}
& 7 & 9 &   &    &    &    &    &    &    & \\
& 1 & 1 & 1 & -1 & -1 & -1 &  0 &  0 &  0 & \\
& 1 & 1 & 1 &  0 &  0 &  0 & -1 & -1 & -1 & \\
& 0 & 1 & 1 & -1 &  0 &  0 & -1 &  0 &  0 & \\
& 1 & 0 & 1 &  0 & -1 &  0 &  0 & -1 &  0 & \\
& 1 & 1 & 0 &  0 &  0 & -1 &  0 &  0 & -1 & \\
& 0 & 1 & 1 &  0 & -1 &  0 &  0 &  0 & -1 & \\
& 1 & 1 & 0 &  0 & -1 &  0 & -1 &  0 &  0 & \\
\end{array}$\\
\hline
  \end{tabular}
\end{center}
Now we call

\begin{center}
{\tt ./rays PROJECT}
\end{center}

which creates the single file

\begin{center}
  \begin{tabular}{|l|}
\hline
    \text{ PROJECT.ray }\\
\hline
  $\begin{array}{rrrrrrrrrrr}& 4 & 9 &&&&&&&&\\ 
  & 0 & 2 & 1 & 2 & 1 & 0 & 1 & 0 & 2 & \\
  & 1 & 2 & 0 & 0 & 1 & 2 & 2 & 0 & 1 & \\
  & 2 & 0 & 1 & 0 & 1 & 2 & 1 & 2 & 0 & \\
  & 1 & 0 & 2 & 2 & 1 & 0 & 0 & 2 & 1 & \\\end{array}$\\
\hline
  \end{tabular}
\end{center}

that corresponds to the four extremal rays of the $3\times 3$ magic
square cone:
\vspace{-0.3cm}
\begin{center}
  \input{magicsqs_ray.pdf_t}
\end{center}
\vspace{-0.4cm}
Every magic square is a nonnegative linear combination of these four
magic squares.

If we turn now to \emph{integer} magic squares, we are looking for a
Hilbert basis of the $3\times 3$ magic square cone. As the default
settings for \File{hilbert} are the same as for \File{rays}, we can
use the same input file
\begin{center}
  \begin{tabular}{|l|}
\hline
  \text{ PROJECT.mat } \\
\hline
$\begin{array}{rrrrrrrrrrr}
& 7 & 9 &   &    &    &    &    &    &    & \\
& 1 & 1 & 1 & -1 & -1 & -1 &  0 &  0 &  0 & \\
& 1 & 1 & 1 &  0 &  0 &  0 & -1 & -1 & -1 & \\
& 0 & 1 & 1 & -1 &  0 &  0 & -1 &  0 &  0 & \\
& 1 & 0 & 1 &  0 & -1 &  0 &  0 & -1 &  0 & \\
& 1 & 1 & 0 &  0 &  0 & -1 &  0 &  0 & -1 & \\
& 0 & 1 & 1 &  0 & -1 &  0 &  0 &  0 & -1 & \\
& 1 & 1 & 0 &  0 & -1 &  0 & -1 &  0 &  0 & \\
\end{array}$\\
\hline
  \end{tabular}
\end{center}
for this computation. However, we call

\begin{center}
{\tt ./hilbert PROJECT}
\end{center}

which creates the single file

\begin{center}
  \begin{tabular}{|l|}
\hline
    \text{ PROJECT.hil }\\
\hline
  $\begin{array}{rrrrrrrrrrr}& 4 & 9 &&&&&&&&\\ 
  & 0 & 2 & 1 & 2 & 1 & 0 & 1 & 0 & 2 & \\
  & 1 & 2 & 0 & 0 & 1 & 2 & 2 & 0 & 1 & \\
  & 2 & 0 & 1 & 0 & 1 & 2 & 1 & 2 & 0 & \\
  & 1 & 0 & 2 & 2 & 1 & 0 & 0 & 2 & 1 & \\
  & 1 & 1 & 1 & 1 & 1 & 1 & 1 & 1 & 1 & \\\end{array}$\\
\hline
  \end{tabular}
\end{center}

that corresponds to the five elements in the minimal Hilbert basis of
the $3\times 3$ magic square cone: 
\vspace{-0.3cm}
\begin{center}
  \input{magicsqs_hil.pdf_t}
\end{center}
\vspace{-0.4cm}
Every integer magic square is a nonnegative \emph{integer} linear
combination of these five integer magic squares. Note that the all-$1$
square is in the interior of the magic square cone.



\subsection{Computing circuits and Graver bases}
Let us compute the homogeneous primitive partition identities of order
$n=4$. Before we do the simple computation, let us explain what a
homogeneous primitive partition identity is.

A \important{partition identity} is any identity of the form
\[
a_1+\ldots+a_k=b_1+\ldots+b_l
\]
with (generally not distinct) integer numbers $0<a_i,b_j\leq n$. It is
called a \important{homogeneous partition identity}, if $k=l$. A homogeneous
partition identity is called \important{primitive} if no proper homogeneous
subidentity exists.

For example,
\[
1+2+3=2+2+2
\]
is a homogeneous partition identity which is not primitive, since it
contains the subidentity
\[
1+3=2+2
\]
which is in fact primitive.

The description of the homogeneous primitive partition identities for
fixed $n$, however, is exactly the description of the Graver basis of
the matrix
\[
A=
\left(
\begin{array}{ccccc}
 1 & 1 & 1 & \ldots & 1 \\
 1 & 2 & 3 & \ldots & n \\
\end{array}
\right).
\]
Let us finally do the computation for $n=4$. We create an input file
\File{hppi4} for \FourTiTwo{} which looks as follows:
\begin{myverbatim}
2 4
1 1 1 1
1 2 3 4
\end{myverbatim}
and invoke \FourTiTwo{} via
\begin{myverbatim}
./graver hppi4
\end{myverbatim}
This call will create an output file \File{hppi4.gra} that looks
like:
\begin{myverbatim}
5 4
-1 2 -1  0
-2 3  0 -1
-1 1  1 -1
 0 1 -2  1
-1 0  3 -2
\end{myverbatim}
Thus, there are $5$ homogeneous primitive partition identities of
order $n=4$:
\begin{eqnarray*}
1+3   & = & 2+2\\
1+1+4 & = & 2+2+2\\
1+4   & = & 2+3\\
2+4   & = & 3+3\\
1+4+4 & = & 3+3+3
\end{eqnarray*}

%%%%%%%%%%%%%%%%%%%%%%%%%%%%%%%%%%%%%%%%%%%%%%%%%%%%%%%%%%%%%%%%%%%%
%
% 4ti2 User Guide, Advanced guide
%
% $Id$
%
%%%%%%%%%%%%%%%%%%%%%%%%%%%%%%%%%%%%%%%%%%%%%%%%%%%%%%%%%%%%%%%%%%%

\chapter{Advanced guide}

In this part, we deal with several more advanced problem
specifications in \FourTiTwo{}.

First we introduce \emph{affine systems} and their encodings. In
fact, affine systems are the basic objects used in \FourTiTwo{},
since every linear system is transformed into an affine system.
However, in the integer situation, it is not always possible to
transform an affine system back into a linear system without adding
variables or modulo constraints.

Next, we demonstrate how one can exploit bounds on integer variables
to truncate the solution set.

Again, we use a few sample problems to demonstrate how to use
\FourTiTwo{}. After working through this part, you should know about
how to run the following functions on affine systems:
\begin{itemize}
\item \File{qsolve}, \File{rays}, \File{circuits}
\item \File{zsolve}, \File{hilbert}, \File{graver}
\item \File{markov}, \File{groebner}, \File{normalform}, \File{minimize}
\end{itemize}


\section{Affine systems and their encodings}

In \FourTiTwo, the definition of an \emph{affine system} is slightly
different for the continuous and for the integer situation. In fact,
the definition for the integer case is simply the integer analogue
to the definition for the continuous case.

\subsection{Continuous affine systems}

Let $a+\Lattice_{\R}$ be a linear affine space given by the vector
$a$ and by generators for the linear space $\Lattice_{\R}$. We wish
to find a finite sign-compatible description for the set of all
vectors $x\in a+\Lattice_{\R}$.

\subsection{Integer affine systems}

Let $a+\Lattice_{\Z}$ be an ``integer linear affine space'' given by
the vector $a\in\Z^n$ and by generators for the lattice
$\Lattice_{\Z}\subseteq\Z^n$. We wish to find a finite
sign-compatible description for the set of all (integer) vectors
$x\in a+\Lattice_{\Z}$.

\subsection{Specifying an affine system in \FourTiTwo{}}

Currently, only homogeneous affine systems can be solved in
\FourTiTwo{} over $\R$ and over $\Z$ using the functions
\File{qsolve} and \File{zsolve}, respectively. In order to call
these functions, one needs to specify an affine system to
\FourTiTwo.

As an example, let consider the linear space $\Lattice_{\R}$ and the
lattice $\Lattice_{\Z}$ both spanned by the two vectors $(1,-2,1,0)$
and $(2,-3,-0,1)$. Moreover, consider the sign-constraints
$(1,2,2,0)$. Thus, we are looking for a finite explicit
sign-compatible description for all $x$ that can be written as
\[
x=\left(
\begin{array}{rr}
1 & 2\\
-2 & -3\\
1 & 0\\
0 & 1\\
\end{array}
\right)\lambda,
\]
with $\lambda\in\R^2$ and $\lambda\in\Z^2$, respectively.

In order to solve this affine system using \File{qsolve} or
\File{zsolve}, we create the following input files to encode the
affine system:
\begin{center}
  \begin{tabular}{|l|l|}
\hline
    \text{ PROJECT.lat } & \text{ PROJECT.sign }\\
\hline
  $\begin{array}{rrrrrr}& 2 & 4 & & &\\& 1 & -1 & 1 & 0 &\\& 2 & -3 & 0 & 1 &\\ \end{array}$ &
  $\begin{array}{rrrrrr}& 1 & 4 & & &\\& 1 &  2 & 2 & 0 &\\ \\ \\\end{array}$\\
\hline
  \end{tabular}
\end{center}
and then call
\begin{center}
{\tt ./qsolve PROJECT}
\end{center}
and
\begin{center}
{\tt ./zsolve PROJECT}
\end{center}
In the continuous case, this creates the files \File{PROJECT.qhom}
and \File{PROJECT.qfree}, and in the integer case this creates the
files \File{PROJECT.zhom} and \File{PROJECT.zfree}.

In contrast to calling \File{qsolve} and \File{zsolve} on a inear
system, these calls create only two files, since at the moment only
homogeneous affine systems (with $a=0$) are supported by
\FourTiTwo{}.


\bibliographystyle{plain}
\bibliography{4ti2_manual}

\end{document}
